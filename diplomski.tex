\documentclass[times, utf8, diplomski]{fer}
\usepackage{booktabs, pdfpages, float}

\begin{document}

% TODO: Navedite broj rada.
\thesisnumber{2350}

% TODO: Navedite naslov rada.
\title{Učinkovito sažimanje genoma korištenjem referentnog genoma}

% TODO: Navedite vaše ime i prezime.
\author{Juraj Radanović}

\maketitle

% Ispis stranice s napomenom o umetanju izvornika rada. Uklonite naredbu \izvornik ako želite izbaciti tu stranicu.
\begin{figure} [H]
	\includegraphics[width=\textwidth, height=\paperheight, keepaspectratio]{izvornik.pdf}
\end{figure}
\thispagestyle{empty}
% Dodavanje zahvale ili prazne stranice. Ako ne želite dodati zahvalu, naredbu ostavite radi prazne stranice.
\zahvala{Zahvaljujem se mentorici doc. dr. sc. Mirjani Domazet-Lošo na vremenu i pomoći u izradi rada}

\tableofcontents

\chapter{Uvod}
Određivanje DNK organizma, tj. sekvenciranje je postalo jeftinije i brže nego prije, zahvaljujući novim tehnologijama sekvenciranja. Jedna od svrha sekvenciranja bi bila detekcija abnormalnih DNK u organizmu koja mogu upućivati na bolest. Sekvenciranje je jedan proces, koji dovodi do pohranjivanja DNK sekvence, prijenosa i procesiranja iste. Svaki od ovih procesa zahtjeva specifične implementacije kako bi svaki proces bio što više optimiziran, memorijski i vremenski. U ovome radu se bavim pohranjivanjem genoma, konkretnije njihovom kompresijom. Što se više genoma sekvencira, to je više podataka za pohranu i treba ih moći pohraniti na efikasne načine u smislu memorije i smanjenja gubitaka podataka tijekom dekompresije. Zapis ljudskog genoma je u redovima $10^{9}$  okteta i držanje takvih datoteka je nemoguće. Stoga je potrebno napraviti algoritam koji će efikasno pohraniti taj genom. Postoje općeniti algoritmi za kompresiju, kao npr. 7zip, ali oni su neučinkoviti za pohranu genoma iz razloga što su genomi srodnih vrsta veoma slični. Konkretno, ljudski DNK su 99.9\% slični, što znači da postoje puno redundantnih podataka. U ovom radu ću implementirati algoritam koji koristi referentni genom uz ciljni genom koji se komprimira, kako bi se iskoristila neka inherentna svojstva genoma i time ostvario veći stupanj kompresije.

\chapter{Zaključak}
Zaključak.

\nocite{*}
\bibliography{literatura}
\bibliographystyle{fer}

\begin{sazetak}
Sažetak na hrvatskom jeziku.

\kljucnerijeci{Ključne riječi, odvojene zarezima.}
\end{sazetak}

% TODO: Navedite naslov na engleskom jeziku.
\engtitle{Title}
\begin{abstract}
Abstract.

\keywords{Keywords.}
\end{abstract}

\end{document}
